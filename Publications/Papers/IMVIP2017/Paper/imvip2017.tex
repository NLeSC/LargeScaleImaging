% ********************************************************************
% *                  Format for IMVIP 2017  papers,                  *
% *         based on the IMVIP 2001, 2006, 2014-2016 templates       *
% ********************************************************************
\documentclass[a4paper,11pt]{article}



\setlength{\topmargin}{-0.5cm}
\setlength{\headsep}{.5cm}
%\setlength{\footskip}{1.0cm}
\setlength{\textheight}{24cm}
\setlength{\textwidth}{17cm}
\setlength{\evensidemargin}{-.5cm}
\setlength{\oddsidemargin}{-.5cm}



\usepackage{fourier}
\usepackage{color}
 \usepackage{graphicx}
\usepackage{url}
\usepackage[affil-it]{authblk}
\usepackage{amsmath}
\usepackage{wrapfig}
\usepackage{xspace}

\usepackage[T1]{fontenc}
\usepackage{times}


\pagestyle{empty}

%%%%
\begin{document}

\title{Shape and Moment Invariants Descriptor for Structured Images}

\author{E. Ranguelova}
\affil{Netherlands eScience Centre\\ Amsterdam, The Netherlands.}
\date{}
\maketitle
\thispagestyle{empty}



\begin{abstract}
Here goes the abstract.
\end{abstract}
\textbf{Keywords:} image mathching, shape descriptors, moment invariants, 



%%%%%%%%%%%%%%%%%%%%%%
\section{Introduction}
\begin{wrapfigure}{r}{0.5\textwidth}
  \vspace{-20pt}
  \begin{center}
    \includegraphics[width=0.4\textwidth, height=0.28\textwidth]{iontas.png}
\end{center}
\vspace{-20pt}
  \caption{Introduction Banner.}
  \vspace{-0pt}
\end{wrapfigure}
%%%%%%%%%%%%%%%%%%%%%%
Here figure for introduction.
%%

\section{Related Work}


\subsection{Salient region detectors}


\subsubsection{Region descriptors}


\section{Shape and Moment Invariant descriptor}

\subsubsection{Simple shape invariants}
\subsubsection{Moment invariants}
\subsubsection{Descriptor}

\paragraph{My Paragraph:} 
This is a paragraph.


\section{Performance Evaluation}

\subsection{Vgg dataset}
Consider the variable $x \in \mathbb{R}$, 
\begin{equation}
f(x)=x^2+2 \label{eq:mylabel}
\end{equation}
Equation (\ref{eq:mylabel}) is a polynomial of order 2.
%% comment line


Figure \ref{fig:myfig} shows individual colour channels. 
\begin{figure}[!ht]
\begin{center}
\includegraphics[width=1\textwidth]{south_campus.jpg}
\end{center}
\vspace{-20pt}
  \caption{A figure}\label{fig:myfig}
  \vspace{-10pt}
\end{figure}



%% comment line

An example of table is shown Table \ref{tab:mytab}.
\begin{table}[!ht]
\begin{center}
\begin{tabular}{|c|c|c|}
\hline
x & y & z\\
\hline
\end{tabular}
\end{center}
\vspace{-20pt}
\caption{Example of table.} \label{tab:mytab}
  \vspace{-10pt}
\end{table}

\subsection{OxFrei dataset}

\section{Conclusion}


\section*{Acknowledgments}

%%\subsection{Bibliographic references}
%%References in a bib file format (e.g. imvip2017.bib given in the template) can be inserted using bibtex along with
%%\LaTeX\xspace/pdflatex (e.g.  \cite{hartley} or  \cite{jain, goodfellow}). 




%%%%%%%%%%%%%%%%%%%%%%%%
\appendix

\section{Vgg dataset matching results }



\section{OxFrei dataset matching results }



\bibliographystyle{apalike}

\bibliography{imvip2017}


\end{document}


% ********************************************************************
% *                  Format for IMVIP 2017  papers,                  *
% *         based on the IMVIP 2001, 2006, 2014-2016 templates       *
% ********************************************************************
\documentclass[a4paper,11pt]{article}



\setlength{\topmargin}{-0.5cm}
\setlength{\headsep}{.5cm}
%\setlength{\footskip}{1.0cm}
\setlength{\textheight}{24cm}
\setlength{\textwidth}{17cm}
\setlength{\evensidemargin}{-.5cm}
\setlength{\oddsidemargin}{-.5cm}



\usepackage{fourier}
\usepackage{color}
\usepackage{graphicx}
 \graphicspath{{./Figs/}}
 \DeclareGraphicsExtensions{.pdf,.jpeg,.png}
\usepackage{url}
\usepackage[affil-it]{authblk}
\usepackage{amsmath}
\usepackage{wrapfig}
\usepackage{xspace}

\usepackage[T1]{fontenc}
\usepackage{times}
\usepackage{verbatim}
\usepackage{subcaption}

\pagestyle{empty}

%%%%
\begin{document}

\title{Shape and Moment Invariants Local Descriptor for Structured Images}

\author{Elena Ranguelova}
\affil{Netherlands eScience Centre\\ Amsterdam, The Netherlands.}
\date{}
\maketitle
\thispagestyle{empty}



\begin{abstract}
Here goes the abstract.Here goes the abstract.Here goes the abstract.Here goes the abstract.Here goes the abstract.
\end{abstract}
\textbf{Keywords:} image mathching, shape descriptors, moment invariants, 



%%%%%%%%%%%%%%%%%%%%%%
\section{Introduction}
 
\begin{wrapfigure}{r}{0.5\textwidth}
 \vspace{-20pt} 
%    \includegraphics[width=0.4\textwidth, height=0.28\textwidth]{iontas.png}
\begin{center}
\begin{subfigure}[b]{0.247\textwidth}
  \includegraphics[width=\textwidth, height=\textwidth]{cool_car_scale5}
\end{subfigure}
\begin{subfigure}[b]{0.247\textwidth}
\includegraphics[width=\textwidth]{cool_car_viewpoint5}
\end{subfigure}
\end{center}
\vspace{-20pt}
\begin{center}
\begin{subfigure}[b]{0.247\textwidth}
  \includegraphics[width=\textwidth]{1graffiti_blur4}
\end{subfigure}
\begin{subfigure}[b]{0.247\textwidth}
\includegraphics[width=\textwidth]{2freiburg_center_blur4}
\end{subfigure}
\end{center}
\vspace{-20pt}
\caption{``Is it the same object or scene?'' Matching two images under different transformation using local interest regions detected by MSER.\\ {\em Top image pair} (scale and viewpoint): SURF descriptor yields false negative (similarity score $0.096$), while the proposed SMI descriptor - true positive ($0.89$).\\
{\em Bottom image pair} (blur): SURF gives false positive ($0.27$), while SMI - true negative ($-0.11$).}
  \vspace{-10pt}
\end{wrapfigure}\label{fig:intro1}
Here goes the Introduciton. Here goes the Introduciton. Here goes the Introduciton. Here goes the Introduciton. Here goes the Introduciton.Here goes the Introduciton. Here goes the Introduciton. Here goes the Introduciton. Here goes the Introduciton. Here goes the Introduciton.Here goes the Introduc
Here goes the Introduciton. Here goes the Introduciton. Here goes the Introduciton. Here goes the Introduciton. Here goes the Introduciton.Here goes the Introduciton. Here goes the Introduciton. Here goes the Introduciton. Here goes the Introduciton. Here goes the Introduciton.Here goes the Introduc

%%%%%%%%%%%%%%%%%%%%%%
Here goes the Introduciton. Here goes the Introduciton. Here goes the Introduciton. Here goes the Introduciton. Here goes the Introduciton.Here goes the Introduciton. Here goes the Introduciton. Here goes the Introduciton. Here goes the Introduciton. Here goes the Introduciton.Here goes the Introduciton. Here goes the Introduciton. Here goes the Introduciton. Here goes the Introduciton. Here goes the Introduciton.Here goes the Introduciton. Here goes the Introduciton. Here goes the Introduciton. Here goes the Introduciton. Here goes the Introduciton.Here goes the Introduciton. Here goes the Introduciton. Here goes the Introduciton. Here goes the Introduciton. Here goes the Introduciton. Here goes the Introduciton. Here goes the Introduciton. Here goes the Introduciton. Here goes the Introduciton. Here goes the Introduciton. Here goes the Introduciton. Here goes the Introduciton. Here goes the Introduciton.
Here goes the Introduciton. Here goes the Introduciton. Here goes the Introduciton. Here goes the Introduciton. Here goes the Introduciton.Here goes the Introduciton. Here goes the Introduciton. Here goes the Introduciton. Here goes the Introduciton. Here goes the Introduciton.Here goes the Introduc
Here goes the Introduciton. Here goes the Introduciton. Here goes the Introduciton. Here goes the Introduciton. Here goes the Introduciton.Here goes the Introduciton. Here goes the Introduciton. Here goes the Introduciton. Here goes the Introduciton. Here goes the Introduciton.Here goes the Introduc
\section{Related Work}

\subsection{Salient region detectors}
Blah, blah blhBlah, blah blhBlah, blah blhBlah, blah blhBlah, blah blhBlah, blah blhBlah, blah blh

Figure \ref{fig:myfig} shows individual colour channels. 
\begin{figure}[!ht]
\begin{center}
\includegraphics[width=1\textwidth]{south_campus.jpg}
\end{center}
\vspace{-20pt}
  \caption{A figure}\label{fig:myfig}
  \vspace{-10pt}
\end{figure}


\subsection{Region descriptors}
{\em State of the art in region descriptor}

\section{Image matching with Shape and Moment Invariant descriptor}

\subsection{Simple shape invariants}
\subsection{Moment invariants}
Consider the variable $x \in \mathbb{R}$, 
\begin{equation}
f(x)=x^2+2 \label{eq:mylabel}
\end{equation}
Equation (\ref{eq:mylabel}) is a polynomial of order 2.
%% comment line

\subsection{Matching}

\paragraph{My Paragraph:} 
This is a paragraph. 

\section{Performance Evaluation}

\subsection{VGG dataset}

%% comment line

The performance results on the VGG dataset are summarized in Table \ref{tab:vgg}.
\begin{table}[!ht]
\begin{center}
\begin{tabular}{|l||c|c|c|c|c|c|c|}

\hline
Det. + descr. & TP & TN & FP & FN & Acc. &Prec. &Recall\\
\hline
\hline
MSER + SURF & $128$ & $428$ &$4$ & $16$ & $0.965$ & $0.969$ & $0.889$\\
\hline
MSER + \bf{SMI} & $122$ &$430$  &$2$  &$22$  &$0.958$  & $0.98$ & $0.847$\\
\hline
BIN + SURF & $122$ & $426$ & $6$ & $22$ & $0.951$ & $0.953$ &$0.847$\\
\hline
BIN (All) + \bf{SMI} &$84$  &$432$  &$0$  &$60$ &$0.89$  & $1$ &$0.58$ \\
\hline
BIN (Largest) + \bf{SMI} &$112$  &$424$  &$8$  &$32$ &$0.93$  & $0.93$ &$0.77$ \\
\hline
\end{tabular}
\end{center}
\vspace{-20pt}
\caption{Performance of salient region detectors and descriptors on the VGG dataset.} \label{tab:vgg}
  \vspace{-10pt}
\end{table}

\subsection{OxFrei dataset}

The performance results on the VGG dataset are summarized in Table \ref{tab:oxfrei}.
\begin{table}[!ht]
\begin{center}
\begin{tabular}{|l||c|c|c|c|c|c|c|}

\hline
Det. + descr. & TP & TN & FP & FN & Acc. &Prec. &Recall\\
\hline
\hline
MSER + SURF & $3309$ & $28848$ & $2904$ & $660$ & $0.90$ & $0.53$ & $0.83$\\
\hline
MSER + \bf{SMI} & $2957$ & $31162$ & $590$ & $1012$ & $0.95$ &$0.83$ & $0.74$\\
\hline
BIN + SURF & $2513$ & $28198$ & $3554$ & $1456$ & $0.85$ &$0.41$ & $0.63$\\
\hline
BIN (All) + \bf{SMI}  & $1275$ & $31298$ & $454$ & $2694$ & $0.91$ &$0.73$ & $0.32$\\
\hline
BIN (Largest) + \bf{SMI}  & $2079$& $28474$ & $3278$ & $1890$ & $0.85$ & $0.38$ & $0.52$\\
\hline
\end{tabular}
\end{center}
\vspace{-20pt}
\caption{Performance of salient region detectors and descriptors on the OxFrei dataset.} \label{tab:oxfrei}
  \vspace{-10pt}
\end{table}

\section{Conclusion}


%\section*{Acknowledgments}

%%\subsection{Bibliographic references}
%%References in a bib file format (e.g. imvip2017.bib given in the template) can be inserted using bibtex along with
%%\LaTeX\xspace/pdflatex (e.g.  \cite{hartley} or  \cite{jain, goodfellow}). 


%%%%%%%%%%%%%%%%%%%%%%%%
\appendix

\section{VGG dataset matching results }



\section{OxFrei dataset matching results }



\bibliographystyle{apalike}

\bibliography{imvip2017}


\end{document}


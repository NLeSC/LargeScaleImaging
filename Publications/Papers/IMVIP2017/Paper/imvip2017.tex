% ********************************************************************
% *                  Format for IMVIP 2017  papers,                  *
% *         based on the IMVIP 2001, 2006, 2014-2016 templates       *
% ********************************************************************
\documentclass[a4paper,11pt]{article}



\setlength{\topmargin}{-0.5cm}
\setlength{\headsep}{.5cm}
%\setlength{\footskip}{1.0cm}
\setlength{\textheight}{24cm}
\setlength{\textwidth}{17cm}
\setlength{\evensidemargin}{-.5cm}
\setlength{\oddsidemargin}{-.5cm}



\usepackage{fourier}
\usepackage{color}
\usepackage{graphicx}
 \graphicspath{{./Figs/}}
 \DeclareGraphicsExtensions{.pdf,.jpeg,.png}
\usepackage{url}
\usepackage[affil-it]{authblk}
\usepackage{amsmath}
\usepackage{wrapfig}
\usepackage{xspace}

\usepackage[T1]{fontenc}
\usepackage{times}
\usepackage{verbatim}
\usepackage{subcaption}

\pagestyle{empty}

%%%%
\begin{document}

\title{Shape and Moment Invariants Local Descriptor for Structured Images}

\author{Elena Ranguelova}
\affil{Netherlands eScience Centre\\ Amsterdam, The Netherlands.}
\date{}
\maketitle
\thispagestyle{empty}



\begin{abstract}
Here goes the abstract.Here goes the abstract.Here goes the abstract.Here goes the abstract.Here goes the abstract.
\end{abstract}
\textbf{Keywords:} image mathching, shape descriptors, moment invariants, 



%%%%%%%%%%%%%%%%%%%%%%
\section{Introduction}
 
\begin{wrapfigure}{r}{0.5\textwidth}
 \vspace{-20pt} 
\begin{center}
\begin{subfigure}[b]{0.247\textwidth}
  \includegraphics[width=\textwidth, height=\textwidth]{cool_car_scale5}
\end{subfigure}
\begin{subfigure}[b]{0.247\textwidth}
\includegraphics[width=\textwidth]{cool_car_viewpoint5}
\end{subfigure}
\end{center}
\vspace{-20pt}
\begin{center}
\begin{subfigure}[b]{0.247\textwidth}
  \includegraphics[width=\textwidth]{1graffiti_blur4}
\end{subfigure}
\begin{subfigure}[b]{0.247\textwidth}
\includegraphics[width=\textwidth]{2freiburg_center_blur4}
\end{subfigure}
\end{center}
\vspace{-20pt}
\caption{``Is it the same object or scene?'' Matching two images under different transformation using local interest regions detected by MSER.\\ {\em Top image pair} (scale and viewpoint): SURF descriptor yields false negative (similarity score $0.096$), while the proposed SMI descriptor - true positive ($0.89$).\\
{\em Bottom image pair} (blur): SURF gives false positive ($0.27$), while SMI - true negative ($-0.11$).}
  \vspace{-10pt}
\end{wrapfigure}\label{fig:intro1}
Here goes the Introduciton. Here goes the Introduciton. Here goes the Introduciton. Here goes the Introduciton. Here goes the Introduciton.Here goes the Introduciton. Here goes the Introduciton. Here goes the Introduciton. Here goes the Introduciton. Here goes the Introduciton.Here goes the Introduc
Here goes the Introduciton. Here goes the Introduciton. Here goes the Introduciton. Here goes the Introduciton. Here goes the Introduciton.Here goes the Introduciton. Here goes the Introduciton. Here goes the Introduciton. Here goes the Introduciton. Here goes the Introduciton.Here goes the Introduc

%%%%%%%%%%%%%%%%%%%%%%
Here goes the Introduciton. Here goes the Introduciton. Here goes the Introduciton. Here goes the Introduciton. Here goes the Introduciton.Here goes the Introduciton. Here goes the Introduciton. Here goes the Introduciton. Here goes the Introduciton. Here goes the Introduciton.Here goes the Introduciton. Here goes the Introduciton. Here goes the Introduciton. Here goes the Introduciton. Here goes the Introduciton.Here goes the Introduciton. Here goes the Introduciton. Here goes the Introduciton. Here goes the Introduciton. Here goes the Introduciton.Here goes the Introduciton. Here goes the Introduciton. Here goes the Introduciton. Here goes the Introduciton. Here goes the Introduciton. Here goes the Introduciton. Here goes the Introduciton. Here goes the Introduciton. Here goes the Introduciton. Here goes the Introduciton. Here goes the Introduciton. Here goes the Introduciton. Here goes the Introduciton.
Here goes the Introduciton. Here goes the Introduciton. Here goes the Introduciton. Here goes the Introduciton. Here goes the Introduciton.Here goes the Introduciton. Here goes the Introduciton. Here goes the Introduciton. Here goes the Introduciton. Here goes the Introduciton.Here goes the Introduc
Here goes the Introduciton. Here goes the Introduciton. Here goes the Introduciton. Here goes the Introduciton. Here goes the Introduciton.Here goes the Introduciton. Here goes the Introduciton. Here goes the Introduciton. Here goes the Introduciton. Here goes the Introduciton.Here goes the Introduc
\section{Related Work}

\subsection{Salient region detectors}
Blah, blah blhBlah, blah blhBlah, blah blhBlah, blah blhBlah, blah blhBlah, blah blhBlah, blah blh


\subsection{Region descriptors}
{\em State of the art in region descriptor}

\section{Image matching with Shape and Moment Invariant descriptor}
We propose a set of several Shape and Moment Invariants (SMI) to encode each salient region into a feature vector (descriptor) used for the region matching. 
The SMI descriptor contains two parts: {\em simple shape invariants}  and {\em moment invariants}.
\begin{equation}
SMI_i = \{S_i, M_i\}
\end{equation} \label{eq:smi}

\subsection{Simple shape invariants}
A binary shape of a region $R_i$ can be described by a set of simple shape properties defined over the original shape or over the equivalent ellipse $E_i$ with seond order moments the same as the region. These properties are: the region's area $a_i$, the area of the region's covex hull $a^c_i$, the length of the major and minor axes of $E_i$, $\mu_i$ and $\nu_i$ and the distance between the foci of the ellipse $\phi_i$.% and the angle between the major axis and the $x$-axis of $E_i$, $\theta_i$. 
From these basic properties, a set of affine invariants are defined in Table \ref{tab:ssi}.   

\begin{table}[!ht]
\begin{center}
\begin{tabular}{|l||l|l|}
\hline
Invariant & Definition & Description\\
\hline
\hline
Relative Area & $\tilde{a}_i = {a_i}/{A}$ & region's area normalized by the image area $A$\\
\hline
Ratio Axes Lengths & $r_i = {\nu_i}/{\mu_i}$& ratio between $E_i$ minor and major axes lengths\\
%\hline
%Orientation & $o_i = \theta_i$& angle between the major axis and the $x$-axis of $E_i$ \\
\hline
Eccentricity &$e_i = \phi_i/\mu_i$& $e_i \in [0,1]$ (0 is a circle, 1 is a line segment.)\\
\hline
Solidity & $s_i = {a_i}/{a_i^c} $ & proportion of the convex hull pixels, that are also in the region. \\
\hline
\end{tabular}
\end{center}
\vspace{-20pt}
\caption{Simple shape invariants.} \label{tab:ssi}
  \vspace{-10pt}
\end{table}
The simple shape invariants part of $SMI_i$ is 
\begin{equation}
S_i = \{\tilde{a}_i, r_i, e_i, s_i\}
\end{equation} \label{eq:ssi}

\subsection{Moment invariants}
Moment invariants are a group of efficient invariant object descriptors. Flusser et al. introduced a general framework for the derivation of moment invariants of any order. \cite{Flusser06momentinvariants}
\subsection{Matching}

\paragraph{My Paragraph:} 
This is a paragraph. 

\section{Performance Evaluation}

\subsection{VGG dataset}

%% comment line

The performance results on the VGG dataset are summarized in Table \ref{tab:vgg}.
\begin{table}[!ht]
\begin{center}
\begin{tabular}{|l||c|c|c|c|c|c|c|}

\hline
Det. + descr. & TP & TN & FP & FN & Acc. &Prec. &Recall\\
\hline
\hline
MSER + SURF & $128$ & $428$ &$4$ & $16$ & $0.965$ & $0.969$ & $0.889$\\
\hline
MSER + \bf{SMI} & $122$ &$430$  &$2$  &$22$  &$0.958$  & $0.98$ & $0.847$\\
\hline
BIN + SURF & $122$ & $426$ & $6$ & $22$ & $0.951$ & $0.953$ &$0.847$\\
\hline
BIN (All) + \bf{SMI} &$84$  &$432$  &$0$  &$60$ &$0.89$  & $1$ &$0.58$ \\
\hline
BIN (Largest) + \bf{SMI} &$112$  &$424$  &$8$  &$32$ &$0.93$  & $0.93$ &$0.77$ \\
\hline
\end{tabular}
\end{center}
\vspace{-20pt}
\caption{Performance of salient region detectors and descriptors on the VGG dataset.} \label{tab:vgg}
  \vspace{-10pt}
\end{table}

\subsection{OxFrei dataset}

The performance results on the VGG dataset are summarized in Table \ref{tab:oxfrei}.
\begin{table}[!ht]
\begin{center}
\begin{tabular}{|l||c|c|c|c|c|c|c|}

\hline
Det. + descr. & TP & TN & FP & FN & Acc. &Prec. &Recall\\
\hline
\hline
MSER + SURF & $3309$ & $28848$ & $2904$ & $660$ & $0.90$ & $0.53$ & $0.83$\\
\hline
MSER + \bf{SMI} & $2957$ & $31162$ & $590$ & $1012$ & $0.95$ &$0.83$ & $0.74$\\
\hline
BIN + SURF & $2513$ & $28198$ & $3554$ & $1456$ & $0.85$ &$0.41$ & $0.63$\\
\hline
BIN (All) + \bf{SMI}  & $1275$ & $31298$ & $454$ & $2694$ & $0.91$ &$0.73$ & $0.32$\\
\hline
BIN (Largest) + \bf{SMI}  & $2079$& $28474$ & $3278$ & $1890$ & $0.85$ & $0.38$ & $0.52$\\
\hline
\end{tabular}
\end{center}
\vspace{-20pt}
\caption{Performance of salient region detectors and descriptors on the OxFrei dataset.} \label{tab:oxfrei}
  \vspace{-10pt}
\end{table}

\section{Conclusion}


%\section*{Acknowledgments}

%%\subsection{Bibliographic references}
%%References in a bib file format (e.g. imvip2017.bib given in the template) can be inserted using bibtex along with
%%\LaTeX\xspace/pdflatex (e.g.  \cite{hartley} or  \cite{jain, goodfellow}). 


%%%%%%%%%%%%%%%%%%%%%%%%
\appendix

\section{VGG dataset matching results }



\section{OxFrei dataset matching results }



\bibliographystyle{apalike}

\bibliography{imvip2017}


\end{document}


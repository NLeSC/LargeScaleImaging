% The ICIP2016 paper according to the given template
%          spconf.sty  - ICASSP/ICIP LaTeX style file, and
%          IEEEbib.bst - IEEE bibliography style file.
% --------------------------------------------------------------------------
\documentclass{article}
\usepackage{spconf,amsmath,graphicx}

% Example definitions.
% --------------------
\def\x{{\mathbf x}}
\def\L{{\cal L}}

% Title.
% ------
\title{Data-driven Region Detector for Structured Image Scenes}
%
% Single address.
% ---------------
\name{Elena Ranguelova\thanks{}}
\address{Netherlands eScience Center\\ Amsterdam, The Netherlands}
%
% For example:
% ------------
%\address{School\\
%	Department\\
%	Address}

%
\begin{document}
%\ninept
%
\maketitle
%
\begin{abstract}
A Data-driven Morphology Salient Regions (DMRS) detector, related to our MSSR detector is proposed.
It demonstrates comparable repeatability to the best-known MSER detector on standard structured scenes and better resolution invariance on a high-resolution benchmark. This is achieved via significantly smaller number of detected regions- a much desired property in the big data era. A data-driven banarization algorithm for a compact binary representation of the image is developed.  Also, new combined dataset for more reliable data and transformation-independent feature detection evaluation is introduced.
While MSER is an excellent detector for generic applications, the DMSR is geared towards the analysis of scientific imagery for detecting precisely semantically meaningful regions. This is a key property in emerging domains such as animal and plant biometrics, with computer vision becoming the vital technology, aiding the wild-life preservation efforts. In this paper, DMSR is demonstrated to better detects the structures in wood microscopy images.

\end{abstract}
%
\begin{keywords}
region detection, data-driven, morhology, structured scences
\end{keywords}
%
\section{Introduction}
\label{sec:intro}

\subsection{Related work}
\label{ssec:relwork}
In \cite{DonoserB06}, an extension of MSER to 3D MSVs is proposed.
\subsection{Contributions}
\label{ssec:contr}


\section{Binary Salient Regions Detection}
\label{sec:binary}


\subsection{Binarization algorithm}
\label{ssec:binarize}


\section{Data-driven Morphology Salient Regions}
\label{sec:DMSR}


\section{Performance  Evaluation}
\label{sec:perf}

\subsection{Standart datasets}
\label{ssec:standart}

\subsubsection{Oxford dataset}
\label{sssec:oxford}
\subsubsection{Freiburg dataset}
\label{sssec:freiburg}
\subsubsection{Combined dataset}
\label{sssec:combined}
\subsubsection{TNT hi-res benchmark}
\label{sssec:tnt}

\section{CONCLUSIONS}
\label{sec:concl}




% To start a new column (but not a new page) and help balance the last-page
% column length use \vfill\pagebreak.
% -------------------------------------------------------------------------
%\vfill
%\pagebreak


\section{REFERENCES}
\label{sec:ref}


% References should be produced using the bibtex program from suitable
% BiBTeX files (here: strings, refs, manuals). The IEEEbib.bst bibliography
% style file from IEEE produces unsorted bibliography list.
% -------------------------------------------------------------------------
\bibliographystyle{IEEEbib}
\bibliography{icip2016}

\end{document}

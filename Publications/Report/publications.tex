\section{Publications}
\label{sec:pubs}


\subsection{Saliency}

In \cite{LiuCVPR2007} the salient object detection is formulated as image segmentation problem. The object is separated from the background on the basis of several features including multi-scale contrast, center-surround histogram and colour spatial distribution for the object description on several levels- locally, regionally and globally. The multi-scale contrast is the local feature, the center-surround histogram is the regional feature and the colour spatial histogram- the global. These features are illustrated on Figure \ref{fig:sal_feat_liu07}.
\begin{figure}[H]
\begin{center}
\includegraphics[width=0.95\textwidth]{fig/SalientFeatures_Liu2007}
\end{center}
\caption{Examples of salient features. From leftto right: input image, multi-scale contrast, center-surround histogram, colour spatial distribution and binary salient mask by CRF.}
\label{fig:sal_feat_liu07}
\end{figure}
A conditional Random Field is trained on these features. 
For the purposes of this research the autors have compiled a large-scale database, MSRA (\cite{msra_db}), presented in section \ref{subsec:msra}. The database is publically available, while the software is not. The proposed methods compared to two other algorithms ``FG" (fuzzy growing) and ``SM" (salient model as computed by the SalientToolbox, described in section \ref{subsec:saltool}). The authors' tends to produce smaller and more focused bounding boxes.

{\em good for automatic cropping?}
In \cite{LCAV-CONF-2009-012} the authors perform a frequency-domain analysis on five stateof-
the-art saliency methods, and compared the spatial frequency
content retained from the original image, which is
then used in the computation of the saliency maps. This
analysis illustrated that the deficiencies of these techniques
arise from the use of an inappropriate range of spatial frequencies.
Based on this analysis, they presented a frequency-tuned
approach of computing saliency in images using low
level features of color and luminance. The resulting saliency maps are better suited to salient object segmentation, with higher precision and
better recall than the analyzed state-of-the-art techniques.

In \cite{YanCVPR2013} the authors address a fundamental problem in saliency detection, namely, the small-scale background structures, which affect the detection. This problem occurs often in natural images. They propose a hierarchical framwork that infers importance values fromimage layers with different scales. The approach is summarized in Figure \ref{fig:hier_yan13}.

\begin{figure}[H]
\begin{center}
\includegraphics[width=0.95\textwidth]{fig/Hierarchy_Yan13}
\end{center}
\caption{An overview of the hierarchical framework. Three image layers are extracted from the input, and then  saliency cues from each of these layers are computed. They are finally fed into a hierarchical model to get the final results.}
\label{fig:hier_yan13}
\end{figure}

For the purpose of their research the authors made a new database available to the community, the Complex Scene Saliency dataset (CSSD) and the Extended CSSD (ECSSD), described in Section \ref{subsec:cssd}. The executable of their software is also available from the project link (\cite{ecssd_db}), but not the source code. The authors report better performence of their method on MSRA-1000 and (E)CSSD datasets comapred to $11$ other state-of-the-art methods.

\subsection{Salient regions}

In \cite{Forssen07} colour extension ot the popular {\em  Maximally Stable Extremal Regions (MSER)} detector is proposed. The author calles his detector {\em Maximally stable colour region (MSCR)}. In comparision on the well-known Visual Geometry Group in Oxford test image sets (\cite{vgg_soft_data}) with known homographies to the original MSER detector, the simple colour MSER extension (MSER3) and a colour blob detector, the MSCR performes better in most cases. The executables of the software for MSCR and blob detectors are available at the author's homepage \cite{forssen07_soft}.

The MSER detectorhas been also extended in 3D to {\em Maximally Stable Volumes (MSVs)} in \cite{DonoserB06}. The MSVs have been used to sucessfully segment 3D medical images and paper fiber networks.

In \cite{Fan08} a structure-guided salient region detector (SGSR) is introduced. It is based on entropy-based saliency theory and shows competitive performance.

In \cite{Wang14} another enhancmenet of MSER is proposal, namely with Canny detector....

\subsection{Convolutional Neural Networks}
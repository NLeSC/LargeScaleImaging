\section{Software}
\label{sec:soft}

The saliency detection, dataset annotation and recognition tools developed by the researchers are often made available to the community.

\subsection{Saliency}
\subsubsection{SaliencyToolbox}\label{subsec:saltool}
The SaliencyToolbox is a collection of Matlab functions and scripts for computing the saliency map for an image, for determining the extent of a proto-object, and for serially scanning the image with the focus of attention. 

\subsubsection{Frequency-tuned Saliency}
The code used in the CVPR 2009 paper ``Frequency-tuned Salient Region Detection'' (\cite{LCAV-CONF-2009-012}) is accessible through the online presentation of the work at \cite{achantaCVPR09}.

\subsection{Dataset annotation}
\subsubsection{LabelMe}\label{subsec:labelme}
LabelMe is a WEB-based image annotation tool that allows researchers to label images and share the annotations with the world. The images canbe organizated into collections, which canbe nested. Images can alsobe uploaded into the system andshared.

The LabelMe MATLAB toolbox is used for interaction with the images and annotations in the LabelMe dataset \ref{subsec:obj_scene_db}. The tool is described in this paper
\cite{Russell2008}. The toolbox also exists in 3D version, LabelMe3D which is described in \cite{Russell2009}. There is also a mobile App version and instructions how could the labelling be outsourced using the Amazon Mechanical Turk.
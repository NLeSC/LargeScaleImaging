\section{Software}
\label{sec:soft}

The saliency detection, dataset annotation and recognition tools developed by the researchers are often made available to the community.

\subsection{Saliency}
\subsubsection{SaliencyToolbox}\label{subsec:saltool}
The SaliencyToolbox is a collection of Matlab functions and scripts for computing the saliency map for an image, for determining the extent of a proto-object, and for serially scanning the image with the focus of attention. 

\subsubsection{Frequency-tuned Saliency}
The code used in the CVPR 2009 paper ``Frequency-tuned Salient Region Detection'' (\cite{LCAV-CONF-2009-012}) is accessible through the online presentation of the work at \cite{achantaCVPR09}.

\subsection{Dataset annotation}
\subsubsection{LabelMe}\label{subsec:labelme}
LabelMe is a WEB-based image annotation tool that allows researchers to label images and share the annotations with the world. The images canbe organizated into collections, which canbe nested. Images can alsobe uploaded into the system andshared.

The LabelMe MATLAB toolbox is used for interaction with the images and annotations in the LabelMe dataset \ref{subsec:obj_scene_db}. The tool is described in this paper
\cite{Russell2008}. The toolbox also exists in 3D version, LabelMe3D which is described in \cite{Russell2009}. There is also a mobile App version and instructions how could the labelling be outsourced using the Amazon Mechanical Turk.

\subsubsection{Photo-identification}


\subsection{Convolutional Neural Networks}
With the excellent performance of CNNs on the ImageNet classification dataset and many other recognition tasks, there is a boom of development of software tools implementing deep leanring and CNNs. 
The tools are often free and open source. At \url{http://deeplearning.net/software_links/} there is an extensive list of such packages/libraries. Here, only the most popular are presented:
\subsubsection{Caffe}
Caffe (\cite{caffe_soft}) is BSD2-Clause license modular framework for deep leanring developed by the Berkeley Vision and Learning Center (BVLC). The framework is a C++ library with Python and MATLAB bindings fr training and deploying general-prurpose CNNs and other deep models on commodity architectures. It is a very popular framework, both in academia and industry due to its speed performance- it can process $60M$ images per day witha single NVIDIA K40 GPU. There is a large community of user and user groups and contributers on GitHub. A technical report for Caffe can be found at it's git repository: \url{https://github.com/BVLC/Caffe/}.
\subsubsection{Torch7}
\subsubsection{Theano}
\subsubsection{DeepLearnToolbox}
\subsubsection{Deeplearning4j}

\section{Conclusions}
\label{sec:conc}

To goal of this document was to present focused review of the state-of-the-art in (large-scale) computer vision research. The reader should have a general overview and find useful pointers (section \underline{\nameref{sec:pubs}}) to where one might direct research and development efforts. It is also important to know the current trends in the field to be able to sustain some level of expertise in it. Based on researching the articles, software, the ever growing number of research image datasets and the (scientific) application domains few main concussions crystallize:

\begin{itemize}
\item{CV field is very large and fast changing. More and more scientific disciplines pose new challenges to CV. To sustain some level of expertise in it, NLeSc should conduct (scientific-driven) CV research, for example within eStep.}
\item{The research efforts should focus around the relevant research questions, presented in the document: localization, identification and classification. Given the current expertise, it is very reasonable to continue improving the MSSR salient region detector (see section \underline{\fullref{subsec:salreg}}). Also, some expertise should be build in Convolutional neural networks (section \underline{\fullref{subsec:cnn}}) by organizing seminars and obtaining practical knowledge for example within project Sherlock.}
\item{The CV researchers in academia are focused mostly on large commercial applications like organizing large photo collections, autonomous driving, etc. There is still not enough effort directed towards the other domain sciences (except from the medical imaging) where NLeSC can contribute (section \underline{\fullref{sec:app}}). Also, sustaining expertise in large scale frameworks for CV systems (section \underline{\fullref{subsec:largescale}}) fits very well the mission and strategy of the center.}
\item{Although not covered by this overview, at NLeSC we see the appearance of new modality, the Point clouds, by which the real world is captured in a way which removes some of the limitations of CV mentioned in section \fullref{sec:intro}. For example in Patty project, methods from CV have been applied to generate PC from images. There is a research in trying to identify objects directly from the PC and it is interesting to find out can the CV algorithms be applied directly on PC or new or adapted algorithms are needed?}

\end{itemize}

